\chapter*{Záver}
\addcontentsline{toc}{chapter}{Záver}

V rámci práce sme vytvorili program TRAVeLer umožňujúci
vizualizáciu sekundárnej štruktúry RNA pomocou existujúceho
obrázka molekuly, ktorý nám slúži ako predloha.

Stromová reprezentácia štruktúry a~následné
použitie \textit{tree-edit-distance} metriky podobnosti dvoch
štruktúr sa ukázali ako správna voľba.
Algoritmus TED nám nie len dá informáciu o tom,
s ako vzdialenými štruktúrami pracujeme, ale dáva nám
aj návod, ako transformovať šablónovú molekulu na tu cieľovú.
U dostatočne podobných štruktúr nám už namapovaná štruktúra
ukáže aké častí sú spoločné a ktoré sa líšia a dá nám
predstavu, ako bude výsledná vizualizácia vyzerať.

Výsledky ukazujú, že ak použijeme štruktúrne dostatočne blízku
molekulu ako šablónu, výsledok vizualizácie je uspokojujúci.
S väčšou vzdialenosťou ale počet prekryvov vo výsledných obrázkoch
stúpa.

V budúcnosti bude vhodné upraviť naše primitívne kresliace algoritmy.
Takýmto vylepšením by bola implementácia otáčania vetiev RNA stromov,
v prípade, že sme našli prekryv.
Druhou možnosťou by bolo pridanie interaktívneho nástroja na úpravu
vzniknutých obrázkov. Ten by umožnil ručne upraviť výsledné vizualizácie.

V našej práci sme sa úplne vyhli pseudouzlom. To nám umožnilo
reprezentovať sekundárnu štruktúru RNA ako usporiadaný, zakorenený strom.
Avšak, pseudouzly sú jej dôležitou súčasťou.
Možným vylepšením by preto bolo, zohľadniť ich existenciu pri mapovaní.
To si ale bude vyžadovať hlbšiu analýzu tohoto problému.

Program bol uvolnený pre používanie biológmi. Budeme očakávať ich reakcie,
na základe ktorých budeme implementovať ďalšie vylepšenia programu.
Dúfame, že aplikácia nájde cestu k svojím používateľom.

