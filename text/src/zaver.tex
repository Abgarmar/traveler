\chapter*{Záver}
\addcontentsline{toc}{chapter}{Záver}

V rámci práce sme vytvorili program TRAVeLer umožňujúci
vizualizáciu sekundárnej štruktúry RNA pomocou existujúceho
obrázka molekuly, ktorý nám slúži ako predloha.

Reprezentovať sekundárnu štruktúru RNA sme sa rozhodli pomocou
stromov. To nám umožnilo použiť \textit{tree-edit-distance}
metriku podobnosti dvoch štruktúr.
Algoritmus TED nám nedáva len informáciu o tom,
s ako vzdialenými štruktúrami pracujeme, ale dáva nám
aj návod, ako transformovať šablónovú molekulu na tu cieľovú.
U dostatočne podobných štruktúr nám už namapovaná štruktúra
dá predstavu, ako bude výsledná vizualizácia vyzerať
a~ukáže, ktoré časti sú v~molekulách spoločné.

Výsledky ukazujú, že ak použijeme štruktúrne dostatočne blízku
molekulu ako šablónu, výsledok vizualizácie bude uspokojujúci.
S väčšou vzdialenosťou ale počet prekryvov vo výsledných obrázkoch
stúpa.

V budúcnosti bude vhodné upraviť naše kresliace algoritmy.
Takýmto vylepšením by bola implementácia otáčania vetiev RNA stromov,
v prípade, že sme našli prekryv.
Druhou možnosťou by bolo pridanie interaktívneho nástroja na úpravu
vzniknutých obrázkov, aby bolo možné výsledné vizualizácie ručne upraviť.
Tým by sa vyriešil problém s prekryvmi, ktoré by si užívateľ sám odstránil.

V našej práci sme sa úplne vyhli pseudouzlom. To nám umožnilo
reprezentovať sekundárnu štruktúru RNA ako usporiadaný, zakorenený strom.
Avšak, pseudouzly sú jej dôležitou súčasťou.
Možným vylepšením by preto bolo, zohľadniť ich existenciu pri mapovaní.
To si ale bude vyžadovať hlbšiu analýzu tohoto problému.

Program bol uvolnený pre používanie biológmi. Budeme očakávať ich reakcie,
na základe ktorých budeme implementovať ďalšie vylepšenia programu.

