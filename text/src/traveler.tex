
\newcommand{\paramI}[1]{\textit{-#1}}
\newcommand{\param}[1]{\textit{\--\--#1}}

\chapter{TRAVeLer - Template RnA Visualization}

V~rámci bakalárskej práce sme vyvinuli konzolovú aplikáciu TRAVeLer.
Program bol vyvíjaný v~C++ a~je určený pre operačné systémy UNIX-ového typu.
Vyvíjaný a~testovaný bol na Linux-e a~FreeBSD.
Podpora ostatných systémov nie je zaručená.





\section{Inštalácia}

Jedinou prerekvizitou k~používaniu našej aplikácie je kompilátor C++
$gcc$ verzie aspoň 4.9.2.

Pri testovaní boli zaznamenané problémy s~regulárnymi výrazmi u~starších verzií
(konkrétne 4.7.2), ktoré ich plne nepodporovali.

Zdrojové kódy programu najnovšej verzie si môžeme stiahnuť pomocou $Git$u príkazom
\begin{code}[frame=none]
# git clone https://github.com/rikiel/bc.git traveler
\end{code}

Preklad zdrojových kódov nám zabezpečí $make$, výsledným spustiteľným programom
bude súbor $traveler/src/build/traveler$.
\begin{code}[frame=none]
  # cd traveler/src && make build
\end{code}





\section{Argumenty programu}

Ak predpokladáme, že program leží na $PATH$, spúšťame ho nasledovne:

\begin{code}[frame=none]
traveler [-h|--help]
traveler [OPTIONS] <TREES>

OPTIONS:
  [<-a|--all> [--overlaps] [--colored] <FILE_OUT>]
  [<-t|--ted> <FILE_MAPPING_OUT>]
  [<-d|--draw> [--overlaps] [--colored] <FILE_MAPPING_IN> <FILE_OUT>]
  [--debug]

TREES:
  <-mt|--match-tree> FILE_FASTA
  <-tt|--template-tree> [--type DOCUMENT_TYPE] DOCUMENT FILE_FASTA
\end{code}

Stručný návod k~programu získame spustením so štandardným
\paramI{h}, alebo \param{help} argumentom.

Prepínače \param{ted} a~\param{draw} sú zjednotené v~\param{all} argumente.
Ich existencia nám umožňuje predpočítať si mapovanie a~následne vygenerovať
aj niekoľko druhov obrázkov. Počítanie TEDu je totiž mnohonásobne
pomalšie ako samotná vizualizácia.

Prepínač \param{match-tree} nám určuje RNA molekulu ktorú ideme vizualizovať.
Ako ďalší parameter očakáva FASTA súbor, ktorého formát uvedieme neskôr.
Šablónu nám určuje prepínač \param{template-tree}.
Kým strom vizualizovanej molekuly sa načítava iba z~fasta súboru,
šablónová molekula potrebuje aj nakreslenie - obrázok z~parametra DOCUMENT.
Podrobnosti ohľadom parametra \param{type} nájdete v~kapitole \nameref{kap:rozsirenie}.

Prepínač \param{overlaps} nám po vygenerovaní obrázku spočíta počet prekryvov
a~vyznačí ich. Ich počet vypíše do samostatného logovacieho súboru, vďaka čomu
rýchlejšie dokážeme identifikovať molekuly, ktoré potrebujú našu zvýšenú
pozornosť.

Prepínač \param{colored} aktivuje farebné zvýrazňovanie použitých operácií
a~\mbox{štruktúrnych} zmien v~strome pri kreslení cieľovej molekuly do obrázka
šablóny. Používame kódovanie farbami z~tabuľky \ref{tab:color_coding}.

\begin{table}
  \centering
  \begin{tabular}{c|c}
    Farba & Význam \\
    \toprule
    \textit{Červená} & vložené bázy (insert) \\
    \midrule
    \textit{Zelená} & premenované bázy (update) \\
    \midrule
    \textit{Modrá} & potrebné prekreslenie pôvodných báz \\
    \midrule
    \textit{Hnedá} & podstromy prekresľovaných multibranch loop \\
    \bottomrule
  \end{tabular}
  \caption{Farebné označenie použitých operácií pri vizualizácií}
  \label{tab:color_coding}
\end{table}

Farbami zvýrazňujeme zmeny v strome, to znamená, že ak napríklad
urobíme $update(AU, AG)$ bázového páru a zmení sa iba jeden nukleotid,
označený bude celý pár ako editovaný.

Modrou označujeme časti, ktoré sme z~nejakého dôvodu potrebovali presunúť a~prekresliť.
Typickým príkladom je prekreslenie loop po vložení alebo zmazaní nejakej bázy.
Vtedy sme kvôli zmene potrebovali prekresliť všetky bázy a~uložiť ich na kružnicu.
Príkladom môže byť obrázok \ref{obr:insert_circle_hairpin} z~predchádzajúcej kapitoly.

Hnedou farbou označujeme celé podstromy multibranch loopy, ktorú potrebujeme
prekresliť. V týchto prípadoch vznikajú často veľké prekryvy a~týmto
sa ich snažíme odlíšiť od ostatných, menej čakaných.
Ak bol ale vrchol pred tým označený, nemeníme jeho farbu (to znamená, že vložený vrchol
bude mať vždy rovnakú farbu, aj keď sme museli prekresliť multibranch loop do ktorej patrí).





\subsection{Formát fasta súboru}

Ako formát súborov kodujúcich stromy používame trochu upravený fasta formát.
Súbor na prvom riadku obsahuje názov molekuly hneď za znakom '$>$' až po prvú medzeru.
Na ďalších riadkoch obsahuje sekvenciu RNA a sekundárnu štruktúru kódovanú
pomocou dot-bracket formátu\footnote{bližšie je popísaný v článku od autorov \citet{DRAWING_COMPARISION}}.
Je ešte zvykom, že riadky sú najviac 80 znakov dlhé.

Fasta súbor pre šablónu potrebuje iba názov a~sekundárnu štruktúru,
pre cieľovú molekulu aj sekvenciu. Je to dané tým, že sekvenciu
si pri šablónovej molekule načítame z~obrázka.



\section{Príklad vstupu}

Teraz uvedieme príklad fasta súboru pre malú podjednotku rRNA človeka z~obrázka
\ref{obr:human_crw} a~ukážeme (jediný) podporovaný formát PostScript obrázka.
Podporujeme iba formát používaný v~CRW databáze (\citet{CRW}).
Ďalšie možné rozšírenia podpory iných formátov rozoberáme v~kapitole
\nameref{kap:rozsirenie}.

\begin{figure}[H]
  \begin{code}[fontsize=\scriptsize, frame=none, samepage=true]
>human
UACCUGGUUGAUCCUGCCAGUAGCAUAUGCUUGUCUCAAAGAUUAAGCCAUGCAUGUCUAAGUACGCACGGCCGGUACAG
UGAAACUGCGAAUGGCUCAUUAAAUCAGUUAUGGUUCCUUUGGUCGCUCGCUCCUCUCCUACUUGGAUAACUGUGGUAAU
UCUAGAGCUAAUACAUGCCGACGGGCGCUGACCCCCUUCGCGGGGGGGAUGCGUGCAUUUAUCAGAUCAAAACCAACCCG
GUCAGCCCCUCUCCGGCCCCGGCCGGGGGGCGGGCGCCGGCGGCUUUGGUGACUCUAGAUAACCUCGGGCCGAUCGCACG
CCCCCCGUGGCGGCGACGACCCAUUCGAACGUCUGCCCUAUCAACUUUCGAUGGUAGUCGCCGUGCCUACCAUGGUGACC
ACGGGUGACGGGGAAUCAGGGUUCGAUUCCGGAGAGGGAGCCUGAGAAACGGCUACCACAUCCAAGGAAGGCAGCAGGCG
CGCAAAUUACCCACUCCCGACCCGGGGAGGUAGUGACGAAAAAUAACAAUACAGGACUCUUUCGAGGCCCUGUAAUUGGA
AUGAGUCCACUUUAAAUCCUUUAACGAGGAUCCAUUGGAGGGCAAGUCUGGUGCCAGCAGCCGCGGUAAUUCCAGCUCCA
AUAGCGUAUAUUAAAGUUGCUGCAGUUAAAAAGCUCGUAGUUGGAUCUUGGGAGCGGGCGGGCGGUCCGCCGCGAGGCGA
GCCACCGCCCGUCCCCGCCCCUUGCCUCUCGGCGCCCCCUCGAUGCUCUUAGCUGAGUGUCCCGCGGGGCCCGAAGCGUU
UACUUUGAAAAAAUUAGAGUGUUCAAAGCAGGCCCGAGCCGCCUGGAUACCGCAGCUAGGAAUAAUGGAAUAGGACCGCG
GUUCUAUUUUGUUGGUUUUCGGAACUGAGGCCAUGAUUAAGAGGGACGGCCGGGGGCAUUCGUAUUGCGCCGCUAGAGGU
GAAAUUCCUUGGACCGGCGCAAGACGGACCAGAGCGAAAGCAUUUGCCAAGAAUGUUUUCAUUAAUCAAGAACGAAAGUC
GGAGGUUCGAAGACGAUCAGAUACCGUCGUAGUUCCGACCAUAAACGAUGCCGACCGGCGAUGCGGCGGCGUUAUUCCCA
UGACCCGCCGGGCAGCUUCCGGGAAACCAAAGUCUUUGGGUUCCGGGGGGAGUAUGGUUGCAAAGCUGAAACUUAAAGGA
AUUGACGGAAGGGCACCACCAGGAGUGGAGCCUGCGGCUUAAUUUGACUCAACACGGGAAACCUCACCCGGCCCGGACAC
GGACAGGAUUGACAGAUUGAUAGCUCUUUCUCGAUUCCGUGGGUGGUGGUGCAUGGCCGUUCUUAGUUGGUGGAGCGAUU
UGUCUGGUUAAUUCCGAUAACGAACGAGACUCUGGCAUGCUAACUAGUUACGCGACCCCCGAGCGGUCGGCGUCCCCCAA
CUUCUUAGAGGGACAAGUGGCGUUCAGCCACCCGAGAUUGAGCAAUAACAGGUCUGUGAUGCCCUUAGAUGUCCGGGGCU
GCACGCGCGCUACACUGACUGGCUCAGCGUGUGCCUACCCUACGCCGGCAGGCGCGGGUAACCCGUUGAACCCCAUUCGU
GAUGGGGAUCGGGGAUUGCAAUUAUUCCCCAUGAACGAGGAAUUCCCAGUAAGUGCGGGUCAUAAGCUUGCGUUGAUUAA
GUCCCUGCCCUUUGUACACACCGCCCGUCGCUACUACCGAUUGGAUGGUUUAGUGAGGCCCUCGGAUCGGCCCCGCCGGG
GUCGGCCCACGGCCCUGGCGGAGCGCUGAGAAGACGGUCGAACUUGACUAUCUAGAGGAAGUAAAAGUCGUAACAAGGUU
UCCGUAGGUGAACCUGCGGAAGGAUCAUUA
...(((((.......))))).((((((((((.(((((((((.....(((.(((..((...(((....((..........)
)...)))))......(((......((((..((..((....(((..................((((....(((((((....
.))))))).....)))).......((((...((((((....))))))...))))....((((((.......(((((.(((
(...((((.((((((((....))))))))..)))).)))).....)))))......))))))...........((((.((
((......))))))))....)))...))))..))))(((..(.(((....((((((((.......)))))))))))....
.))))...((((((((....))))...))))))).((((((..........)))))).((((....))))...)))))).
).....(.(((...(((((...))))).)))).)).))))))....(((((((((((((....))).))))))).)))..
....(((.(((.......)))).)).........((((((.......((((.....((....)).......)))))))))
).))))))))))..........(((.......((((...(((.......(((.(((((((((((((.((((....)))).
...))))))))..)))))))).......((((.(((((...(((((((......)))))))....)))))))))......
................................................................................
..........................................(((((((((..(((((((((..((((((((...(((..
....))).......))))))))..))....(..((....)))))))))).))))).))))...)))...))))....(((
(((...((...((((.........))))...))))))))..........((((((.(((..((((((((.(((((....)
))))))))))))..)))...((....))...)))....))).)))..(((.....((((....))))....)))......
..(((((.(((((((..((..(((((((((((((((((....((((........))))........(((((((....(((
((........((((((........))))))......)))))...((.((((..(((((((((...(((((((((....))
)..((((......))))..)))))).....((((.(((.((((..((((....(((..((((....)))).)))....))
))..)))))))..((((((((.....))))))))....))))...)))).)))...).))))))).....)))))))...
)).))))))))))...(((((((.....(((.......((..((((....))))..)).....))).....)))))))..
....(...((((((((........))))))))...).....))))).....((((((((.......))))))))......
))...)))))))))).))....((.((.(.((((((((.((.((((((((((((..(((((((((((((((.((((((((
((((.....))))))))))))...)))))))))))))))..))))))))))))).)))))))))..).))..))....((
((((((((....))))))))))........
  \end{code}
  \caption{Príklad fasta súboru}
  \label{obr:human_fasta}
\end{figure}

\begin{figure}[H]
\begin{code}[fontsize=\scriptsize, frame=none, samepage=true]
%!
/lwline {newpath moveto lineto stroke} def
  ...
434.00 -129.00 422.00 -138.00 lwline
0.00 setlinewidth
446.00 -421.00 446.00 -412.00 lwline
306.00 -283.00 306.00 -273.00 lwline
  ...
(U) 303.30 -273.00 lwstring
(A) 303.30 -265.00 lwstring
(C) 303.30 -257.00 lwstring
(C) 303.50 -248.68 lwstring
(U) 311.24 -246.68 lwstring
(G) 318.99 -244.68 lwstring
  ...
\end{code}
\caption{Príklad podporovaného formátu post script súboru}
\label{obr:ps_format}
\end{figure}

Kvôli dĺžke PostScript súboru sme uviedli iba jeho časť na obrázku \ref{obr:ps_format}.
V~súbore ignorujeme všetky riadky ktoré sú iného formátu ako
\\
\textit{(B) X Y lwstring},
\\
Kde $B$ označuje bázu a~$X$ a $Y$ sú súradnice bodu, kde je báza nakreslená.
Vďaka tomu, že sú riadky vypisované v~smere $5' \to 3'$, tak zo súboru
vieme určiť primárnu štruktúru RNA.




\section{Výstupne súbory}

Program generuje dva druhy výstupov. Prvým je tabuľka mapovania stromov
ako výstup TED algoritmu. Druhým sú obrázky vo formátoch SVG a PS.

\subsection{Mapovacia tabuľka}

Formát súboru s~mapovacou tabuľkou je nasledovný. Prvý riadok obsahuje
\mbox{\textit{DISTANCE: n}}, kde \textit{n} je editačná vzdialenosť
medzi stromami.
Ďalšie riadky sú vo formáte \mbox{$i$ $j$}, kde \mbox{$i, j$} sú rôzne a~väčšie ako 0
a~ich význam je nasledovný.
Ak \mbox{$i = 0$}, potom do stromu šablóny vkladáme $j$-tý vrchol%
\footnote{Používame post order poradie vrcholov v~strome} cieľovej molekuly.
naopak ak \mbox{$j = 0$}, potom $i$-tý vrchol zo šablóny mažeme. V~ostatných prípadoch
mapujeme $i$-tý vrchol na $j$-tý, tzn. robíme \mbox{$update(i, j)$}.
Príklad časti súboru z~mapovania medzi človekom a~žabou (K03432 a X04025)
je na obrázku \ref{obr:mapping_format}.
Ako vidíme, editačná vzdialenosť je 58, čiže sme 58 nukleotidov pridali alebo zmazali.
Vkladáme nukleotidy s~indexom (v~cieľovej molekule) 356, 365,
nukleotidy číslo 1, 2 v~oboch molekulách mapujeme na seba a~bázy
s~indexom (v~molekule šablóny) 155, 156 mažeme.

\begin{figure}
\begin{code}[fontsize=\scriptsize, frame=none, samepage=true]
DISTANCE: 58
0 356
0 365
  ...
1 1
2 2
  ...
155 0
156 0
  ...
\end{code}
\caption{Časť výstupného súboru s mapovacou tabuľkou}
\label{obr:mapping_format}
\end{figure}




\subsection{PostScript obrázok}

PostScript súbor je zložený z~hlavičky, v~ktorej sú definície kresliacich funkcií za
ktorými sú riadky kreslenia molekuly. Príklad je na obrázku \ref{obr:ps_out}.

Najprv definujeme operácie kreslenia v hlavičke súboru - $lwline$,
$lwstring$ a~$lwarc$ - kreslenie čiar, textu a~kružníc a~ďalšie 
parametre dokumentu, za ktorými nasleduje samotné kreslenie molekuly.

\begin{figure}
\begin{code}[fontsize=\scriptsize, frame=none, samepage=true]
%!
/lwline {newpath moveto lineto stroke} def
/lwstring {moveto show} def
/lwarc {newpath gsave translate scale /rad exch def /ang1 exch def
  /ang2 exch def 0.0 0.0 rad ang1 ang2 arc stroke grestore} def
/Helvetica findfont 8.00 scalefont setfont
0.80 0.80 scale
337.29 1647.24 translate
(G)            -238.24        -721.38        lwstring       
(G)            -243.9         -727.04        lwstring       
(G)            -249.56        -732.7         lwstring       
(G)            -255.21        -738.36        lwstring       
(C)            -260.87        -744.01        lwstring       
(C)            -263.91        -752.09        lwstring       
(C)            -271.44        -756.35        lwstring       
(C)            -280.03        -755.17        lwstring       
(C)            -286.14        -749.04        lwstring       
(G)            -287.29        -740.46        lwstring       
(C)            -283.01        -732.93        lwstring       
(G)            -275.01        -729.87        lwstring       
-260.698       -738.182       -269.182       -729.698        lwline
(C)            -269.36        -724.21        lwstring       
-255.038       -732.532       -263.532       -724.038        lwline
(C)            -263.7         -718.56        lwstring       
-249.388       -726.872       -257.872       -718.388        lwline
(C)            -258.04        -712.9         lwstring       
-243.728       -721.212       -252.212       -712.728        lwline
(C)            -252.38        -707.24        lwstring       
  ...
(5')           -229.75        -712.89        lwstring
(3')           -243.89        -698.75        lwstring
-232.412       -715.552       -229.583       -712.723        lwline
-246.552       -701.412       -243.723       -698.583        lwline
showpage
\end{code}
\caption{Príklad výstupného PostScript súboru}
\label{obr:ps_out}
\end{figure}



\newcommand{\tagt}[1]{\mbox{$<$\textit{#1}$>$}}

\subsection{SVG obrázok}

Podobne funguje kreslenie v SVG súbore, ktorého príklad je na obrázku \ref{obr:svg_out}.
Elementy \tagt{text} vypisujú na danú pozíciu text, \tagt{line}
naopak kreslia čiary a~\tagt{circle} zase kružnice.
Argumenty týchto elementov sú bližšie vysvetlené v~nejakom SVG tutorále
(napríklad online tutoriál od autora \citet{SVG_TUTORIAL}).


\begin{figure}
\begin{code}[fontsize=\scriptsize, frame=none, samepage=true]
<svg xmlns="http://www.w3.org/2000/svg" xmlns:xlink="http://www.w3.org/1999/xlink"
  width="1133.333333" height="1466.666667" viewBox="0 0 1139.172822px 1450.347571px"
  style="
    font-size: 8px; 
    stroke: none; 
    font-family: Helvetica; ">
  <text 
    x="517.486977"
    y="603.524781"
    style="
      stroke: rgb(0, 255, 0); ">5'</text>
  <line 
    x1="681.175823"
    y1="650.435118" 
    x2="681.175823"
    y2="662.435118"
    style="
      stroke: rgb(0, 0, 0); 
      stroke-width: 2; "/>
  <circle 
    cx="616.350806"
    cy="427.616196"
    r="6.276645"
    style="
      stroke: rgb(0, 0, 0); 
      fill: none; "/>
  ...
</svg>
\end{code}
\caption{Príklad výstupného SVG súboru}
\label{obr:svg_out}
\end{figure}





\section{Rozšírenie podpory formátov vstupných obrázkov}
\label{kap:rozsirenie}

Ako sme už uviedli, momentálne podporujeme iba jediný vstupný formát obrázkov.
Je ním PostScript formát používaný databázou CRW od autorov \citet{CRW}.

Pri tvorbe aplikácie sme mysleli na budúcnosť a~tak načítavanie súboru
robíme v~jednom, ľahko rozšíriteľnom module.
Ak potrebujeme pracovať s~inými vstupnými súbormi, naimplementujeme extractor
(viď definícia) a~parametrom \param{type} ho môžeme použiť.
Predvoleným a~jediným implementovaným je práve PostScript extractor
fungujúci nad súbormi z~CRW databázy.

\begin{definice}
  Extractor je objekt, ktorý vie pracovať so súbormi určitého typu
  a~vie z~nich vyňať potrebné položky reprezentujúce sekvenciu RNA
  a~pozície jej báz v~obrázku.
\end{definice}

\renewcommand{\tagt}[1]{\mbox{\textit{#1}}}

Vytvorenie nového typu spočíva v~implementácií rozhrania \tagt{extractor}
a~jeho metódu \tagt{init}, ktorá ako jediný parameter dostáva názov súboru.
Druhou úlohou je pridať dvojicu \tagt{názov, extractor} do tabuľky
implementovaných v~metóde \tagt{create\_extractors}.
Pri spustení programu s~parametrom \param{type názov} použijeme
novoimplementovaný extractor.


