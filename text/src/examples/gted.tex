
\newcommand{\wi}{0.2\hsize}
\newcommand{\wit}{0.5\hsize}
\newcommand{\subtree}[1]{\begin{tikzpicture}[
          on grid,
          font=\tiny,
          scale = \scale,
          node distance = 0.6 cm,
          sibling distance = 0.8 cm,
          text width = 0.2cm,
          level distance = 1.1 cm,
          every node/.style = {scale = \scale, circle, draw, align=center},
          tree/.style = {draw = none, fill = none}]#1\end{tikzpicture}}

\newcolumntype{C}{>{\centering}m{2cm}}

\begin{figure}
    \begin{minipage}{\wi}
      \begin{tikzpicture}[
          on grid,
          font=\large,
          scale = \scale,
          level distance = 1.5 cm,
          every node/.style = {scale = \scale, circle, draw},
          tree/.style = {draw = none, fill = none}]

          \node[tree] (F) {F:};
          \node[right = of F] {4}
          child {
            node {2}
            child {
              node {1}
            }
          }
          child {
            node {3}
          }
          ;
      \end{tikzpicture}
    \end{minipage}
    \begin{minipage}{\wi}
      \begin{tikzpicture}[
          on grid,
          font=\large,
          scale = \scale,
          level distance = 1.5 cm,
          invisible/.style={opacity=0},
          every node/.style = {scale = \scale, circle, draw},
          tree/.style = {draw = none, fill = none}]

          \node[tree] (G) {G:};
          \node (3) [right = of G] {C}
          child {
            node {A}
            % nieje viditelny; je pridavany iba kvoli tomu, aby obidva stromy boli rovnako vysoke
            child[invisible] {
              node {I}
            }
          }
          child {
            node {B}
          }
          ;
      \end{tikzpicture}
    \end{minipage}
    \begin{minipage}{\wit}
      \begin{tabular}{c|c|c|c}
        \mc{$ForestDistance:$} & \mc{\subtree{\node{A};}} & \mc{\subtree{\node (A) {A}; \node[right = of A] {B};}} & \mc{\subtree{\node {C} child {node{A}} child {node{B}};}} \\
        \toprule
        \subtree{\node{1};}                                                 & 0 & 1 & 2 \\
        \midrule
        \subtree{\node{2} child {node{1}};}                                 & 1 & 2 & 1 \\
        \midrule
        \subtree{\node(2){2} child {node{1}}; \node[right = of 2]{3};}      & 2 & 1 & 2 \\
        \midrule
        \subtree{\node{4} child{node{2} child {node{1}}} child {node{3}};}  & 3 & 2 & 1 \\
        \bottomrule
      \end{tabular}
    \end{minipage}
  \caption{Príklad výpočtu GTED medzi stromami $F$ a $G$}
  \label{obr:gted_priklad}
\end{figure}


